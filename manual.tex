\documentclass{article}
\usepackage{amsthm}
\usepackage{amsmath}%,txfonts,cmbright}
\usepackage{enumerate}
\usepackage{amsfonts}
\usepackage{amssymb}
\usepackage{latexsym}
\usepackage[latin1]{inputenc}
\usepackage{graphicx}
\usepackage{tikz}
\usepackage[left=3cm, right=2cm, top=2.5cm, bottom=2.5cm]{geometry}
%\usepackage{fancyhdr}
\usepackage{color}
%\usepackage{lettrine}
\usepackage{calc,pifont}
\usepackage[noindentafter,calcwidth]{titlesec}
%\usepackage[Sonny]{fncychap}
\usepackage{multirow}

\renewcommand{\thefigure}{\thechapter.\arabic{figure}}

\setlength{\headheight}{13pt}

%%%%%%%%%%%%%%%%%%%%%%%%%%%%%%%%%%%%%%%%%%%%%%%%%%%%%%%%%%%%%%%%%%%%%%%
%%%%%%%%%%%%%%%%%% FORMATOS PARA LAS SECCIONES Y SUBSECCIONES %%%%%%%%%
%%%%%%%%%%%%%%%%%%%%%%%%%%%%%%%%%%%%%%%%%%%%%%%%%%%%%%%%%%%%%%%%%%%%%%%
%\titleformat{\section}[display]
%{\Large\filright\sffamily\bfseries}{\titleline[1]{}}{0ex}{\thesection
%\hspace{.3cm}}
%\titlespacing{\section}{2pc}{*2}{*4}

%\newcommand{\Section}[1]
%{
%	\addtocounter{section}{1}
%	\section*{#1}
%	\fancyhead[LO]{\sf Section \thesection. #1}
%	\addcontentsline{toc}{section}{\thesection \hspace{.3cm} #1}
%	\setcounter{subsection}{0}
%}

%\titleformat{\subsection}[display]
%{\large\filright\sffamily} {\titleline[l]{}}{0ex}{\thesubsection \hspace{.3cm}}
%\titlespacing{\subsection}{2pc}{*2}{*4}

%\newcommand{\Subsection}[1]
%{
%	\addtocounter{subsection}{1}
%	\subsection*{#1}
%	\addcontentsline{toc}{subsection}{\thesubsection \hspace{.3cm} #1}
%	\setcounter{subsubsection}{0}
%}

%\titleformat{\subsubsection}[display]
%{\large\filright\sffamily} {\titleline[l]{}}{0ex}{\thesubsubsection \hspace{.3cm}}
%\titlespacing{\subsubsection}{2pc}{*2}{*4}

%\newcommand{\Subsubsection}[1]
%{
%	\addtocounter{subsubsection}{1}
%	\subsubsection*{#1}
%	\addcontentsline{toc}{subsubsection}{\thesubsubsection \hspace{.3cm} #1}
%}
%%%%%%%%%%%%%%%%%%%%%%%%%%%%%%%%%%%%%%%%%%%%%%%%%%%%%%%%%%%%%%%%%%%%%%%
%%%%%%%%%%%%%%%%%%%%%%%%%%%%%%%%%%%%%%%%%%%%%%%%%%%%%%%%%%%%%%%%%%%%%%%
%%%%%%%%%%%%%%%%%%%%%%%%%%%%%%%%%%%%%%%%%%%%%%%%%%%%%%%%%%%%%%%%%%%%%%%



%%%%%%%%%%%%%%%%%%%%%%%%%%%%%%%%%%%%%%%%%%%%%%%%%%%%%%%%%%%%%%%%%%%%%%%
%%%%%%%%%%%%% FORMATOS PARA LOS TEOREMAS, LEMAS, ETC %%%%%%%%%%%%%%%%%%
%%%%%%%%%%%%%%%%%%%%%%%%%%%%%%%%%%%%%%%%%%%%%%%%%%%%%%%%%%%%%%%%%%%%%%%
\newtheorem{Theorem}{Theorem} 
\newtheorem{Definition}[Theorem]{Definition} 
\newtheorem{Lemma}[Theorem]{Lemma} 
\newtheorem{Corollary}[Theorem]{Corollary} 
\newtheorem{Conjecture}[Theorem]{Conjecture} 
\newtheorem{Remark}[Theorem]{Remark} 
\newtheorem{Example}[Theorem]{Example\/} 
\newtheorem{Algorithm}[Theorem]{Algorithm} 
\newtheorem{Proposition}[Theorem]{Proposition} 
\newtheorem{Claim}[Theorem]{Claim} 
\newtheorem{Problem}[Theorem]{Problem}
%%%%%%%%%%%%%%%%%%%%%%%%%%%%%%%%%%%%%%%%%%%%%%%%%%%%%%%%%%%%%%%%%%%%%%%
\newcommand{\Code}[1]
{
	\begin{center}
		\tikz \draw (0,0) node[fill=black!15,text width=13cm]
		{\texttt{#1}};
	\end{center}
}
\renewcommand{\familydefault}{\sfdefault} 
\renewcommand{\sfdefault}{cmbr}
\newcommand{\Ker}{\textsf{Ker}}
%%%%%%%%%%%%%%%%%%%%%%%%%%%%%%%%%%%%%%%%%%%%%%%%%%%%%%%%%%%%%%%%%%%%%%%
%%%%%%%%%%%%%%%%%%%%%%%%%%%%%%%%%%%%%%%%%%%%%%%%%%%%%%%%%%%%%%%%%%%%%%%
%%%%%%%%%%%%%%%%%%%%%%%%%%%%%%%%%%%%%%%%%%%%%%%%%%%%%%%%%%%%%%%%%%%%%%%

\makeindex

\title{{\it CSandPile 0.5's Manual}}
\author{Carlos A. Alfaro \& Carlos E. Valencia}
\date{}

\begin{document}
\maketitle

This document describes the version $0.5$ of the program {\it CSandPile}, a program that allows to compute the sandpile group of a symmetric multidigraph $G$.
The {\it CSandPile} program was programmed in C++ language using the GNU Compiler Collection and had been used over Windows XP, MAC OS 10.5.8 and Ubuntu Linux 9.04.

%We developed a computer program called {\it CSandPile}, because we want to understand the 
%combinatorial structure of the recurrent configurations of the sandpile group of a symmetric multidigraph  $G$, an important part of this thesis.
%This program is oriented to explote the combinatorial structure of the sandpile group of $G$.
The {\it CSandPile} program version $0.5$ is mainly a tool for compute the group operations of the recurrent representatives of non-negative configurations of $G$, 
that is, is the first effort to find the combinatorial structure of the group operations of the recurrent configurations that generate the sandpile group of $G$.

If $c$ be a non-negative configuration, then using {\it CSandPile}, one may compute the following:
\begin{itemize}
	\item the stabilization of $c$,
	\item the recurrent representative of $c$,
	\item the powers of  a recurrent configuration,
	\item the representative  of the inverse of a recurrent configuration $c$,
	\item the recurrent representative of the identity of the sandpile group of $G$, 
	\item the determinant of the Laplacian matrix of $G$,
	\item the powers of the representative of the canonical base.	
\end{itemize}

\section{The structure of {\it CSandPile}}

{\it CSandPile}consists of the file \texttt{csandpile.exe} the executable file if you are using a Windows environment, or \texttt{csandpile.out} the executable file if you are using a UNIX environment.
Also, you need to have an input file \texttt{<my project>.gph}  with the input data.

\subsection{The input file}
The input file \texttt{<my project>.gph}  is structured as follows: 
The first line contains the order of the Laplacian matrix of $G$, the next lines contain the rows of the Laplacian matrix of $G$, 
the next line contain the vertex of $G$ that will play the role of the sink, 
and finally the last line contains some configuration. 
\begin{Example}\label{exa1}
Let $G$ be the cycle $C_4$ with four vertices, therefore the Laplacian matrix of $C_4$ is given by:
\[
	L(C_4)=
	\left[
	\begin{array}{cccc}
		2 & -1 & 0 & -1\\
		-1 & 2 & -1 & 0\\
		0 & -1 & 2 & -1\\
		-1 & 0 & -1 & 2\\ 
	\end{array}
	\right]
\]

The file called ``c4.gph" has the order of the Laplacian matrix of $C_4$, the Laplacian matrix of $C_4$, the vertex $4$ as a sink and the vector $(2,2,1,0)$ as the configuration.
\Code{4 \\
2 -1 0 -1\\
-1 2 -1 0\\
0 -1 2 -1\\
-1 0 -1 2\\
4\\
2 2 1 0\\
}

If we do not write a configuration, the configuration $\sigma_{MAX}=(\deg(1)-1, \deg(2)-1,...,\deg(n)-1)$ will be taken by default. 
%In the before example, the configuration $(1,1,1,1)$ would be taken.
\end{Example}

\subsection{Running {\it CSandPile}}

The files \texttt{csandpile.exe} and \texttt{csandpile.out} are the executables files in Windows and UNIX, respectively. 
You must run it from the console of your operating system. 
The syntax for calling {\it CSandPile} is
\[
\texttt{csandpile <my project> [-option]}
\]
where -option can be one of the following options: 
\Code{%Write, after of the executable file, one of the following options:\\
-s \hspace{1.63cm} to obtain the stable configuration\\
-p \hspace{1.64cm} to obtain the powers of the recurrent configuration\\
-i \hspace{1.65cm} to obtain the identity\\
-r \hspace{1.66cm} to obtain the recurrent configuration\\
-ri \hspace{1.49cm} to obtain the inverse recurrent configuration\\
-det \hspace{1.33cm} to obtain the determinant of the reduced Laplacian matrix\\
-group \hspace{.94cm} to obtain the powers of the standard base\\
-complete n \hspace{.1cm} to create the Laplacian matrix of the complete graph of n\\
\hspace{2.2cm} vertices\\
-path n \hspace{.9cm} to create the Laplacian matrix of the path of n vertices\\
-cycle n \hspace{.74cm} to create the Laplacian matrix of the cycle of n vertices\\
}

When you type and execute \texttt{csandpile} or \texttt{./csandpile.out} if you are working in a UNIX environment on the console, 
the program creates a file called \texttt{<my project>.csp}. %that its contains the following information:

%To execute an option you need to type the executable file followed by the option that you want.
For instance, if you want to obtain the stable configuration, you need to type 
\[
\texttt{csandpile <my project> -s} \text{ or } \texttt{./csandpile.out <my project> -s}. 
\]
and {\it CSandPile} will create a file called \texttt{<my project>.csp}.
Thus, using ``c4.gph" as an input file, the file \texttt{c4.csp} contains the following information:
%For instance, if you executed the option to obtain the stable configuration, the \texttt{csandpile.csp} file contains

\Code{The stable configuration of\\
2 2 1 S\\
is\\
0 1 1 S\\
}
Note that we put a \texttt{S} in the coordinate of the sink.


To obtain the powers of a recurrent configuration, you need to type \texttt{-p} as option. 
Using ``c4.gph" as an input file, the \texttt{c4.csp} file contains

\Code{Checking configuration:  0 1 1 S\\
Powers\\
1 -  0 1 1 S\\
2 -  1 1 1 S\\
3 -  1 1 0 S\\
4 -  1 0 1 S\\
}

To obtain the identity configuration, you need to type \texttt{-i} as an option. 
Using ``c4.gph" as an input file, the \texttt{c4.csp} file contains

\Code{Identity:  1 0 1 S\\
}

To obtain the recurrent configuration of the configuration given in the \texttt{<my project>.gph}, you need to type \texttt{-r} as option. 
Using ``c4.gph" as an input file, the \texttt{c4.csp} file contains

\Code{The recurrent configuration of  2 2 1 S\\
is 0 1 1 S\\
}

To obtain the recurrent inverse configuration of the configuration given in the \texttt{<my project>.gph}, you need to type \texttt{-ri} as option. 
Using ``c4.gph" as an input file, the file \texttt{c4.csp} contains

\Code{Inverse recurrent configuration:  1 1 0 S\\
}

To obtain the determinant of the reduced Laplacian matrix, you need to type \texttt{-det} as option. 
Using ``c4.gph" as an input file, the \texttt{c4.csp} file contains

\Code{Determinant: 4\\
}

To obtain the powers of the canonical base, you need to type \texttt{-group} as option. 
Using ``c4.gph" as an input file, the \texttt{c4.csp} file contains

\Code{Generator 1:  1 0 0 S\\
 Checking configuration:  0 1 1 S\\
 Powers\\
1 -  0 1 1 S\\
2 -  1 1 1 S\\
3 -  1 1 0 S\\
4 -  1 0 1 S\\
\medskip
 Generator 2:  0 1 0 S\\
 Checking configuration:  1 1 1 S\\
 Powers\\
1 -  1 1 1 S\\
2 -  1 0 1 S\\
\medskip
 Generator 3:  0 0 1 S\\
 Checking configuration:  1 1 0 S\\
 Powers\\
1 -  1 1 0 S\\
2 -  1 1 1 S\\
3 -  0 1 1 S\\
4 -  1 0 1 S\\
}

\subsection{Some special graphs}

Also, {\it CSandPile}  can generate the Laplacian matrix of the complete graph of $n$ vertices, 
the path of $n$ vertices and the cycle of $n$ vertices and write it in the \texttt{<my project>.gph} file.
This can be done by typing \texttt{-complete \textit{n}}, \texttt{-cycle \textit{n}} or \texttt{-path \textit{n}} as an option, respectively.
For instance, if you write 
\[
\texttt{csandpile k4 -complete 4}
\]
you will obtain the Laplacian matrix of the complete graph of 4 vertices in the \texttt{k4.gph} file.

\Code{4\\
3 -1 -1 -1\\
-1 3 -1 -1\\
-1 -1 3 -1\\
-1 -1 -1 3\\
4\\
}

Note that  by default, {\it CSandPile} will define the vertex \textit{n} as the sink.
\end{document}
